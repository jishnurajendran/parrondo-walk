\documentclass[10pt,a4paper]{article}
\usepackage{amsmath}
\usepackage{amsfonts}
\usepackage{amssymb}
\usepackage{graphicx}
\usepackage{float}
\usepackage{hyperref}
\usepackage[left=2cm,right=2cm,top=2cm,bottom=2cm]{geometry}
\begin{document}
\begin{flushleft}
Dear Editor,
\end{flushleft}

We thank you for sending us the comments on the paper. We have revised our paper and are resubmitting it. Please find below a summary of changes made and a detailed response to all the comments and qeries of the referees. 
\section*{Summary of changes made}
\begin{enumerate}
\item All the typos mentioned by all Referees and especially pointed out by the first referee have been corrected.

\item We have added three new figures, Figs.~5,6,7. Fig.~5 investigates whether the Parrondo effect survives a change in initial state from two orthogonal states, say $|10\rangle$ to identical 2 coin states: $|00\rangle$ and $|11\rangle$. We conclude from the results of Fig.~5 that initial states have a large bearing on whether we see a Parrondo's paradox or not. Fig.~6, on the other hand investigates the role of the shift operator. A change in shift operator from two wait states (see Eq.~6 of manuscript) to a single wait state (Eq.~7 of manuscript) leads to a significant difference. The asymmetric shift operator (Eq.~7) makes any two coin initial state leads to a Parrondo's paradox. Finally, in Fig.~7, we see what are the implications for three coin initial state. We see similar results to the two coin state.

\item Section IV on Discussion is enlarged to explain the reasons behind seeing the Parrondo's paradox. We explain the new figures 5,6 and 7 and try to understand possible reasons behind observing the Parrondo's paradox especially focussing on shift operator, initial states, entanglement and finally .

\end{enumerate}

\section*{Response to comments and queries of the referees}
\subsection*{Referee 1}
We thank Referee 1 for the encouraging comments and for the responding that "New quantum walks are of great interest to the community as their investigation may lead to new quantum algorithms, which the field search for at present". In the revised manuscript in section 5, conclusion, we add this sentence verbative.
\begin{enumerate}
\item \textbf{Some captions are in regular font, one it is italics, and one is is
slant fond. Please regularize this to all be in PRE format.}\\
All the captions are rewritten in PRE format  
\item \textbf{Please check all sentences carefully, as some are missing a full
stop. Eq. 1 should end with a full stop and not a comma. Put a full stop at the end of equations when they end a sentence.}
Comma is replaced with full stop in Eq.1. Adequate changes are made throughout the manuscript.
\item \textbf{Carefully check the formatting of all your brackets. For example, $\mid 1\rangle$ should be $\vert 1\rangle$ without the redundant space. There are many instances of this throughout the paper.}\\
All the bra-ket notations are corrected, without any redundant space.
\item \textbf{In the caption of fig 2, q=4 should be $q=4$. Check whole paper for
this kind of error.}\\
q=4 is replaced with $q=4$ and similar corrections are also made.
\item \textbf{In three lines below Eqn 6, you have used a full stop as a
multiplication symbol.}\\
$\backslash$cdot is used for all the multiplication symbol, now throughout the revised manuscript.
\item \textbf{The correct LaTeX is not J.M.R. Parrondo but J.~M.~R.~Parrondo . Do
this for all names in the paper including the references. No first names should be in reference [4]. They should be written as follows: M.~Li, Y.-S.~Zhang, and G.-C.~Guo. Remove first names from all other references.}\\
All the names are rewritten in the suggested format. All the first names are removed from the reference.
\end{enumerate}

\subsection*{Referee 2}
We thank the referee for diligently going through our manuscript. We disagree with the referee on his/her conclusion. We respond to his/her comments in detail below.
\begin{enumerate}
\item \textbf{Essentially, all they report is that they have found a pair of coin
operators for which Parrondo-like behavior is observed numerically
over 800 time steps when used with a four dimensional coin instead of
the usual two dimensional coin. This graph does look like this will be
the asymptotic behavior, but there are other instances (e.g., dynamic
percolation) where the asymptotic localization doesn't appear until
around 2000 time steps, so I would count this as only suggestive and
requiring more work to confirm.}\\
It is not correct that we numerically observed it over 800 time steps. We numerically observe in fact over 1600 time steps, which in position space is $-1600$ to $+1600$ which is over $3200$ position spaces.
In the revised version we extended the numerical calculation from 1600 to 2000 steps for fig.[3] (a) and (c) and we could see that the asymptotic nature is still valid in the range of 2000 steps.
\item \textbf{There is some study of the behavior
with different coin initial states, but this is not fully general (the
coin always has two of the four possible components and a phase of pi
between them).}\\
We have extended our study to understand the reasons behind the Parrondo's paradox with two coins and not with a single coin. We studied the influence of initial state, shift operator, entanglement and coherence. We see that shift operator, entanglement or coherence has no or a trivial role in the implementation of a Parrondo's paradox but the orthogonality of the initial two coin state has a major role in the paradoxical behavior obtained, i.e, symmetric coins $\vert 00 \rangle$ and $\vert 11 \rangle$ do not give any paradoxical behavior but the $\vert 10\rangle$ and $\vert 01 \rangle$ initial states lead to Parrondo's paradox.
\item \textbf{The paper is also contains a significant density of typos. Figure 1 is
not necessary to explain the simple division of left and right for winning or losing the game.}\\
Adequate changes and corrections are made in the manuscript for the typos. Fig.1 is a pictorical representation of win and lose defined using $P_R-P_L$.
 
\item \textbf{No insight is given as to why the four dimensional coin
has different behavior, nor whether a three dimensional coin will also
work, with one wait state rather than two. Is this one of many ways to
obtain a Parrondo's paradox with higher dimensional coins, or is it a
rare example?}\\
Further studies on the initial state $\vert 00\rangle$ and $\vert 11\rangle$ gave us results which are similar to that of a single coin initial state. This behavior suggests that the orthogonality in the two coin state is the most plausible reason why we obtain a genuine Parrondo's paradox.\\ 
In response to the query, "is this the only way to obtain a genuine Parrondo's paradox using two coin initial state", we show in Fig.3(a),(c) two distinct cases, one in which two coin state is entangled other in which the are unentangled. More over in Fig.4 also we see over a range of values of $\theta$, where $\theta$ determines the superposition between the two orthogonal initial states $\vert 01 \rangle$ and $\vert 10 \rangle$ we see a genuine Parrondo's paradox. We extended the study to coherent two coin initial state but here in also we do not get a genuine Parrondo's paradox. With 3 coin initial state $***************$ 
\item  \textbf{The choice of $P_R - P_L$ instead of average
position is not properly justified, and it is not made clear which
measure is used in prior work.} \\
In the previous works, Refs. \cite{chandru}-\cite{minli} $P_R-P_L$ is used to define a win or a loss. To maintain continuity as well as to make the comparisons explicit between ours and previous works, we choose to stick with the definitions of median $P_R-P_L$ of the probability distributions as the arbiter of win or loss. Thus in our work too a player is considered winner if the probability of finding the particle to the right of the origin $P_R$ is greater than the probability to the left of the origin $P_L$, that is, $P_R > P_L$ in the asymptotic limit of the quantum walk. However if $P_L > P_R$ in the asymptotic limits of the quantum walks then the player losses.
\item \textbf{Citations are missing, e.g., Meyer \& Blumer, or not updated to published versions.}\\
We have added the citation to  Meyer \& Blumer in the revised manuscript. All the references are corrected and updated to the published version.
\end{enumerate}

\subsection*{Referee 3}
We thank referee 3 for his insightful comments and suggestions, however we disagree with his conclusion. The detailed response to his/her comments given below.  
\begin{enumerate}
\item \textbf{I agree with the second referee in that the paper does not give any
clue on why two coins are necessary to obtain the effect. I do not
share, however, the concerns of the second referee about the
stationarity of the effect. In my opinion, plots in Fig. 3 depict a
clear stationary behavior.}\\
We have extended our manuscript to include two state coins with initial state $\vert 00\rangle$ and $\vert 11\rangle$. These results are similar to that of a single coin initial state. This behavior suggests that the orthogonality in the initial two coin state is the most plausible reason why we obtain a genuine Parrondo's paradox. We thank the referee for accepting that with 1600 time steps we get asymptotic behavior. Anyway we have included a 2000 time step simulation too.  
\item \textbf{I also agree with the referee that the use of $P_R-P_L$ is not
justified. The authors claim that "expectation value may be positive
but $P_R-P_L$ maybe negative this would be and absurd result " (sic).
This happens with classical random variables as well. There is nothing
absurd in a random variable $X$ with positive average and negative
median (i.e., $P(X>0)$ smaller than $P(X<0)$ ). One can choose either the
median or the average as a definition of "winning" and "losing" games,
but both criteria are reasonable.}\\
We agree with this comment. As suggested one can choose median or average as a definition for winning or losing. In the previous works Refs. \cite{chandru,minli} $P_R-P_L$ is used to define a win or a loss. We continue with this criteria in our work too. A player is considered winner if the probability of finding the particle to the right of the origin $P_R$ is greater than the probability to the left of the origin $P_L$, that is, $P_R > P_L$ in the asymptotic limit of the quantum walk. However if $P_L > P_R$ in the asymptotic limits of the quantum walks then the player losses.

\item \textbf{I miss as well a comparison between the corresponding Markovian random
walk, i.e., the random walk resulting from collapsing the wave function
in each step. It would be interesting to compare the Markovian and
non-Markovian dynamics of the random walk.}\\
The random walk devised from collapsing the wave function at each step is similar to that of a classical random walk. When the wave function is collapsed and position is measured and the evolution is applied on this collapsed state this behaves exactly like a classical random walk. Further details on clasical vs quantum random wals can be seen from Chapter 3 of \cite{portugal}. "When we measure the particle position after the first step, we
destroy the correlations between different positions, which are typical of quantum
systems. If we do not measure and apply the coin operator followed by the shift
operator successively, the quantum correlations between different positions can have
constructive or destructive interference, effectively generating a behavior different
from the classical context, which is a characteristic of quantum walks."(Section 3.2, page no.25, third paragraph, $5^{th}$ line).
 

\end{enumerate}

\begin{thebibliography}{99}
\bibitem{chandru}
C.~M.~Chandrashekar, S.~Banerjee, Parrondo's game using a discrete-time quantum walk, Physics Lett. A 375 (2011) 1553.

\bibitem{minli}
M.~Li, Y.~S.~Zhang, G.-C.~Guo, Quantum Parrondo's games constructed by quantum random walk, arXiv:quant-ph/1303.6831 (2013).

\bibitem{portugal}
R.~Portugal, Quantum Walks and Search Algorithms, Springer DOI 10.1007/978-1-4614-6336-8
\end{thebibliography}

\end{document}